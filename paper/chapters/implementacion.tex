\chapter{Implementación}\label{ch:implementacion}

Para este trabajo, se decidió experimentar sobre distintas implementaciones para un mismo tipo de datos abstracto.
El tipo elegido, una pila, es representado por una referencia a un nodo que contiene un valor y una referencia al nodo que le sigue (que se encuentra debajo del nodo en la pila).
Para el último nodo de la pila, la referencia al nodo siguiente será denominada como nula.
El código de esta representación se presenta en la figura \ref{fig:node-rep} con sus variantes de implementación IO y STM.

\begin{figure}[!h]
  \centering
  \begin{minted}[linenos,breaklines,fontsize=\footnotesize]{haskell}
data NodeIO a = NdIO { val :: a, next :: IORef (NodeIO a) } | Null
data NodeSTM a = NdSTM { val :: a, next :: TVar (NodeSTM a) } | Null
  \end{minted}
  \caption{Representación de un nodo en Haskell sobre IO y sobre STM}
  \label{fig:node-rep}
\end{figure}

Las operaciones a realizar sobre una pila son las de apilar (\mintinline{haskell}{push}) y desapilar (\mintinline{haskell}{pop}).
Todas las variantes de implementación que realicemos cumplen la misma interfaz para estas funciones, presentada en la figura \ref{fig:interfazStack}.

\begin{figure}[!h]
  \centering
  \begin{minted}[linenos,breaklines,fontsize=\footnotesize]{haskell}
push :: Eq a => Stack a -> a -> IO ()
pop :: Eq a => Stack a -> IO a
  \end{minted}
  \caption{Interfaz para las funciones \mintinline{haskell}{push} y \mintinline{haskell}{pop}}
  \label{fig:interfazStack}
\end{figure}

La función \mintinline{haskell}{push} inserta un nuevo nodo al tope de la pila con un valor recibido por parámetro. Por el otro lado, la función pop lee el valor que se encuentra en el tope de la pila y remueve el nodo, retornando el valor contenido en el nodo como resultado. En el caso que no haya ningún elemento en la pila, la ejecución deberá arrojar una excepción.

En este capítulo se encuentran detalladas las distintas implementaciones utilizadas para la evaluación de performance: \mintinline{haskell}{LockFreeStack} (sección \ref{sec:lfs}), \mintinline{haskell}{EliminationBackoffStack} (sección \ref{sec:ebs}), y \mintinline{haskell}{StackSTM} (sección: \ref{sec:stackstm}). Las primeras dos estructuras son maneras conocidas para la implementación de una pila concurrente y se encuentran detalladas en \cite{shavit} que hacen uso de la primitiva de sincronización \emph{compare and set}. La estructura restante es una implementación \emph{naive} de una pila concurrente utilizando las herramientas provistas la librería STM de Haskell.

En el código utilizado para la experimentación, cada versión de \mintinline{haskell}{push} y \mintinline{haskell}{pop} es nombrada de manera que se puede identificar a qué implementación pertenece. Por ejemplo, las funciones de un \mintinline{haskell}{LockFreeStack} implementado sobre IO tienen los nombres \mintinline{haskell}{pushLFSIO} y \mintinline{haskell}{popLFSIO}.

\section{LockFreeStack}\label{sec:lfs}
La estructura de datos \mintinline{haskell}{LockFreeStack} (LFS), también conocida como TreiberStack \cite{treiber}, consiste de una pila concurrente libre de locks implementada utilizando algoritmos optimistas. Sus algoritmos consisten en intentar realizar la operación sin tener que pausar la ejecución para esperar un recurso y si hay una falla en el intento, reintentar. En este caso, los algoritmos realizan un backoff previo a realizar un reintento.

La representación de un \mintinline{haskell}{LockFreeStack} consiste de una referencia al tope de la pila y una estructura auxiliar de backoff que será la encargada de determinar la cantidad de tiempo que un hilo debe esperar antes de reintentar.

\begin{figure}[!h]
  \centering
  \begin{minted}[breaklines,fontsize=\footnotesize]{haskell}
data LockFreeStackIO a = LFSIO { top :: IORef (NodeIO a), backoffLFS :: Backoff }
pushLFSIO :: Eq a => LockFreeStackIO a -> a -> IO ()
popLFSIO :: Eq a => LockFreeStackIO a -> IO a

data LockFreeStackSTM a = LFSSTM { top :: TVar (NodeSTM a), backoffLFS :: Backoff }
pushLFSSTM :: Eq a => LockFreeStackSTM a -> a -> IO ()
popLFSSTM :: Eq a => LockFreeStackSTM a -> IO a
  \end{minted}
  \caption{Representación de un LFS en sus implementaciones IO y STM y sus operaciones}
  \label{repLFS}
\end{figure}

\begin{figure}[!h]
  \begin{subfigure}[b]{0.5\textwidth}
  \begin{minted}[linenos,breaklines,fontsize=\footnotesize]{haskell}
pushLFSIO lfs value = do
  ret <- newIORef True
  node <- newNodeIO value
  whileM_ (readIORef ret) $ do
    b <- tryPushIO lfs node
    if b
      then writeIORef ret False
      else backoff $ backoffLFS lfs
  \end{minted}
  \caption{Código \mintinline{haskell}{pushLFS} sobre IO}
  \end{subfigure}
  \begin{subfigure}[b]{0.4\textwidth}
  \begin{minted}[linenos,breaklines,fontsize=\footnotesize]{haskell}
pushLFSSTM lfs value = do
  ret <- newIORef True
  node <- newNodeSTM value
  whileM_ (readIORef ret) $ do
    b <- tryPushSTM lfs node
    if b
      then writeIORef ret False
      else backoff $ backoffLFS lfs
  \end{minted}
  \caption{Código \mintinline{haskell}{pushLFS} sobre STM}
  \end{subfigure}
\par\bigskip
  \begin{subfigure}[b]{0.5\textwidth}
  \begin{minted}[linenos,breaklines,fontsize=\footnotesize]{haskell}
popLFSIO lfs = do
  ret <- newIORef True
  res <- newIORef Nothing

  whileM_ (readIORef ret) $ do
    returnNode <- tryPopIO lfs
    if returnNode /= Null
      then do
        writeIORef res $ Just (val returnNode)
        writeIORef ret False
      else backoff (backoffLFS lfs)

  readIORef res >>= return.fromJust
  \end{minted}
  \caption{Código \mintinline{haskell}{popLFS} sobre IO}
  \end{subfigure}
  \begin{subfigure}[b]{0.4\textwidth}
  \begin{minted}[linenos,breaklines,fontsize=\footnotesize]{haskell}
popLFSSTM lfs = do
  ret <- newIORef True
  res <- newIORef Nothing

  whileM_ (readIORef ret) $ do
    returnNode <- tryPopSTM lfs
    if returnNode /= Null
      then do
        writeIORef res $ Just (val returnNode)
        writeIORef ret False
      else backoff (backoffLFS lfs)

  readIORef res >>= return.fromJust
  \end{minted}
  \caption{Código \mintinline{haskell}{popLFS} sobre STM}
  \end{subfigure}
  \caption{Funciones de \mintinline{haskell}{LockFreeStack}}
  \label{fig:lfs-push-pop}
\end{figure}

La idea detrás de los algoritmos es que cada hilo de ejecución intente realizar una operación de escritura o lectura sobre la pila. Si el hilo no es exitoso en el intento, debe volver a intentar luego de esperar una cantidad de tiempo determinada por una estructura de backoff. Este comportamiento se puede apreciar en la figura~\ref{fig:lfs-push-pop} donde se muestra el código de las funciones \mintinline{haskell}{pushLFSIO} y \mintinline{haskell}{popLFSIO}. En ellas, se encuentran las llamadas a las funciones \mintinline{haskell}{tryPushIO} y \mintinline{haskell}{tryPopIO} respectivamente que serán detalladas en la subsección~\ref{sub:lfs-intentos}, así también la llamada a la función \mintinline{haskell}{backoff} que se detalla en la subsección~\ref{sub:backoff}.

\subsection{Backoff}\label{sub:backoff}
En la implementación, la estructura de backoff es del tipo de backoff exponencial. Al realizar la operación de backoff, la estructura toma un número aleatorio sobre un rango entre 0 y el límite establecido por parámetro y paraliza la ejecución del thread por esa cantidad de milisegundos. Luego la estructura duplica el valor del límite para la próxima vez que se realice la operación. Esta estructura de datos también se encuentra presentada en \cite{shavit}.

\begin{figure}[!h]
  \begin{minted}[linenos,breaklines,fontsize=\footnotesize]{haskell}
data Backoff = BCK {minDelay :: Int, maxDelay :: Int, limit :: IORef Int}

newBackoff :: Int -> Int -> IO Backoff
newBackoff min max = (newIORef min) >>= return.(BCK min max)

backoff :: Backoff -> IO ()
backoff b = do
  backoffLimit <- readIORef $ limit b
  delayInMilliseconds <- randomRIO (0, backoffLimit)
  writeIORef (limit b) (min (maxDelay b) (2 * backoffLimit))
  threadDelay (toMicroseconds delayInMilliseconds)
    where toMicroseconds = (*) 1000
  \end{minted}
  \caption{Estructura de backoff}
  \label{fig:backoff-code}
\end{figure}

La especificación de la función \mintinline{haskell}{backoff} según \cite{shavit} establece que el número generado aleatoriamente sea la cantidad de milisegundos que debe esperar el hilo de ejecución. Por eso, es necesario convertir esa cantidad a microsegundos, tal como aparece en la última linea de la figura~\ref{fig:backoff-code}, ya que la función \mintinline{haskell}{threadDelay} de Haskell toma microsegundos como parámetro \cite{threadDelay}.

\subsection{Intento de escritura o lectura}\label{sub:lfs-intentos}
El intento de escritura o lectura se ve implementado en las funciones \mintinline{haskell}{tryPush} y \mintinline{haskell}{tryPop} respectivamente. Ambas siguen la misma idea: observar el valor del tope de la pila y luego ejecutar CAS comparando lo que observa actualmente con lo observado en primer lugar. Para el caso del intento de apilar, se retorna el resultado de la ejecución de CAS, y para el intento de desapilar se retorna \mintinline{haskell}{Null} en el caso de fracaso, o el valor leído del tope de la pila en caso de éxito.

En el caso particular en la que un hilo desee remover un objeto en la pila, se debe realizar una verificación que la pila no se encuentre vacía. Si no hay ningún objeto para remover, la ejecución arroja una excepción \mintinline{haskell}{EmptyException}.

\begin{figure}[!h]\label{intentos-types}
  \begin{minted}[linenos,breaklines,fontsize=\footnotesize]{haskell}
tryPushIO :: Eq a => LockFreeStackIO a -> NodeIO a -> IO Bool
tryPopIO :: Eq a => LockFreeStackIO a -> IO (NodeIO a)

tryPushSTM :: Eq a => LockFreeStackSTM a -> NodeSTM a -> IO Bool
tryPopSTM :: Eq a => LockFreeStackSTM a -> IO (NodeSTM a)
  \end{minted}
  \caption{Tipo de las funciones de intento de lectura y escritura en sus variantes IO y STM}
\end{figure}

\begin{figure}[!h]\label{intentos-code}
  \begin{subfigure}[b]{0.5\textwidth}
  \begin{minted}[linenos,breaklines,fontsize=\footnotesize]{haskell}
tryPushIO lfs node = do
  oldTop <- readIORef (top lfs)
  writeIORef (next node) oldTop
  atomCASIO (top lfs) oldTop node
  \end{minted}
  \caption{Código \mintinline{haskell}{tryPushIO}}
  \end{subfigure}
  \begin{subfigure}[b]{0.5\textwidth}
    \begin{minted}[linenos,breaklines,fontsize=\footnotesize]{haskell}
tryPushSTM lfs node = do
  oldTop <- atomically $ readTVar (top lfs)
  atomically $ writeTVar (next node) oldTop
  atomCASSTM (top lfs) oldTop node
    \end{minted}
    \caption{Código \mintinline{haskell}{tryPushSTM}}
\end{subfigure}
\end{figure}

\begin{figure}[!h]
  \begin{subfigure}[b]{0.55\textwidth}
    \begin{minted}[linenos,breaklines,fontsize=\footnotesize]{haskell}
tryPopIO lfs = do
  oldTop <- readIORef (top lfs)
  if oldTop == Null
    then
      throw EmptyException
    else do
      newTop <- readIORef (next oldTop)
      b <- atomCASIO (top lfs) oldTop newTop
      if b
        then return oldTop
        else return Null
    \end{minted}
    \caption{Código \mintinline{haskell}{tryPopIO}}
  \end{subfigure}
  \begin{subfigure}[b]{0.6\textwidth}
  \begin{minted}[linenos,breaklines,fontsize=\footnotesize]{haskell}
tryPopSTM lfs = do
  oldTop <- atomically $ readTVar (top lfs)
  if oldTop == Null
    then
      throw EmptyException
    else do
      newTop <- atomically $ readTVar (next oldTop)
      b <- atomCASSTM (top lfs) oldTop newTop
      if b
        then return oldTop
        else return Null
  \end{minted}
  \caption{Código{} \mintinline{haskell}{tryPopSTM}}
  \end{subfigure}
  \caption{Variantes de implementación para los intentos de escritura y lectura de un LFS}
\end{figure}


\section{EliminationBackoffStack}\label{sec:ebs}

Un \mintinline{haskell}{EliminationBackoffStack} (EBS) es una extensión para un LFS cuya idea principal es aprovechar el tiempo previo a un reintento para intentar un intercambio con otro hilo que esté esperando para ejecutar la operación inversa. Es decir, un hilo lector intentará realizar un intercambio con un hilo escritor y vice versa.

Para lograr este objetivo, se necesita de tres estructuras de datos auxiliares: \mintinline{haskell}{LockFreeExchanger}, \mintinline{haskell}{EliminationArray}, y \mintinline{haskell}{RangePolicy}. La representación de estas estructuras en sus variantes IO y STM se ve reflejada en la figura \ref{fig:ebs-reps}

La estructura EBS fue propuesta en \cite{ebs} como una alternativa de mejor escalabilidad al LFS y luego mencionada con ejemplos de código Java en \cite{shavit}. Estos últimos ejemplos de código son la base para la implementación que se detalla a continuación.

\begin{figure}[!h]
  \begin{minted}[linenos,breaklines,fontsize=\footnotesize]{haskell}
data EliminationBackoffStackIO a = EBSIO { top :: IORef (NodeIO a), capacity :: Int, eliminationArray :: EliminationArrayIO a, policy :: TLS RangePolicy}
data EliminationArrayIO a = EAIO {exchanger :: [LockFreeExchangerIO a], duration :: Integer}
data LockFreeExchangerIO a = LFEIO {slot :: IORef (Maybe a, State)}

data EliminationBackoffStackSTM a = EBSSTM {top :: TVar (NodeSTM a), capacity :: Int, eliminationArray :: EliminationArraySTM a, policy :: TLS RangePolicy}
data EliminationArraySTM a = EASTM {exchanger :: [LockFreeExchangerSTM a], duration :: Integer}
data LockFreeExchangerSTM a = LFESTM {slot :: TVar (Maybe a, State)}
\end{minted}
\caption{Variantes de implementación para las estructuras necesarias para un EBS}
\label{fig:ebs-reps}
\end{figure}

\subsection{LockFreeExchanger}\label{sub-lfe}
El \mintinline{haskell}{LockFreeExchanger} (LFE) consiste de una referencia que guarda información de estado (tipo de datos \mintinline{haskell}{State}) y un valor a intercambiar.
El valor a intercambiar es representado por el tipo \mintinline{haskell}{Maybe} para poder representar el caso en el que el LFE esté vacío y no haya hilos tratando de realizar un intercambio.
El estado puede tomar 3 posibles valores:

\begin{figure}[!h]
  \centering
\begin{minted}[linenos,breaklines,fontsize=\footnotesize]{haskell}
data State = EMPTY | WAITING | BUSY
\end{minted}
  \caption{Tipo de datos \mintinline{haskell}{State}}
  \label{fig:state}
\end{figure}

\begin{itemize}
\item{\mintinline{haskell}{EMPTY}: no hay ningún valor asociado a la referencia}
\item{\mintinline{haskell}{WAITING}: un hilo ha escrito un valor en la referencia y está esperando a que un hilo lo lea}
\item{\mintinline{haskell}{BUSY}: un hilo a leído el valor que se encuentra en la referencia y escribe un valor para intercambiar}
\end{itemize}

La única función que realiza esta estructura de datos es la operación \mintinline{haskell}{exchange}. El método consiste en leer el estado del LFE y realizar la acción correspondiente al estado. La operación debe ocurrir dentro de una ventana de tiempo que es establecida mediante un parámetro \mintinline{haskell}{timeout} al invocar la función.

\begin{figure}[!h]
  \centering
  \begin{minted}[linenos,breaklines,fontsize=\footnotesize]{haskell}
exchangeIO :: (Eq a) => LockFreeExchanger a -> Maybe a -> Integer -> IO (Maybe a)
exchangeSTM :: (Eq a) => LockFreeExchanger a -> Maybe a -> Integer -> IO (Maybe a)
  \end{minted}
  \caption{Tipado de la función exchange en sus variantes IO y STM}
  \label{fig:exchange-types}
\end{figure}

Para la implementación de la función \mintinline{haskell}{exchange} utilizamos una función auxiliar \mintinline{haskell}{getSlot} para simular el comportamiento de la operación \mintinline{java}{get} de la clase \mintinline{java}{AtomicStampedReference}  \cite{asr-java}, ya que la componente \mintinline{haskell}{slot} de un LFE tiene este tipo en el código Java que se encuentra en \cite{shavit}.
La función toma una referencia a una tupla de tipo \mintinline{haskell}{(Maybe a,State)} (\mintinline{haskell}{slot}) y una referencia \mintinline{haskell}{IORef} a un valor de tipo \mintinline{haskell}{State} (\mintinline{haskell}{stampHolder}).
La función \mintinline{haskell}{getSlot} se encarga de leer la tupla de la referencia \mintinline{haskell}{slot} para luego escribir su estado en la referencia \mintinline{haskell}{stampHolder} y retornar el valor de la tupla como resultado de la función.
Los tipos de la función en sus variantes IO y STM en la figura y sus código se encuentran en las figuras \ref{fig:getSlot-types} y \ref{fig:getSlot-code} respectivamente.

\begin{figure}[!h]
  \centering
  \begin{minted}[linenos,breaklines,fontsize=\footnotesize]{haskell}
getSlotIO :: IORef (Maybe a, State) -> IORef State -> IO (Maybe a)
getSlotSTM :: TVar (Maybe a, State) -> IORef State -> IO (Maybe a)
  \end{minted}
  \caption{Tipos de las funciones \mintinline{haskell}{getSlot} en sus variantes IO y STM}
  \label{fig:getSlot-types}
\end{figure}

\begin{figure}[!h]
  \centering
    \begin{subfigure}[b]{0.4\textwidth}
    \begin{minted}[linenos,breaklines,fontsize=\footnotesize]{haskell}
getSlotIO slot stampHolder = do
  (val, state) <- readIORef slot
  writeIORef stampHolder state
  return val
    \end{minted}
  \end{subfigure}
  \begin{subfigure}[b]{0.5\textwidth}
    \begin{minted}[linenos,breaklines,fontsize=\footnotesize]{haskell}
getSlotSTM slot stampHolder = do
  (val, state) <- atomically $ readTVar slot
  writeIORef stampHolder state
  return val
    \end{minted}
  \end{subfigure}
  \caption{Código para \mintinline{haskell}{getSlot} en sus variantes IO y STM}
  \label{fig:getSlot-code}
\end{figure}

A continuación se lista el comportamiento del algoritmo según los distintos casos del estado que tome el LFE. En cada ítem se notarán las líneas de código de la figura \ref{fig:exchange-io} en las que se puede apreciar el comportamiento detallado.

\begin{itemize}
\item{
\mintinline{haskell}{EMPTY}: el hilo actual intenta escribir el valor con una llamada a \mintinline{haskell}{atomCAS} el valor y el estado pasa a \mintinline{haskell}{WATING} en caso de éxito. Si la llamada a \mintinline{haskell}{atomCAS} falla (retorna \mintinline{haskell}{False}), se vuelve al inicio del ciclo en la línea 9.

Si la llamada a \mintinline{haskell}{atomCAS} es exitosa, se espera a que el estado pase a \mintinline{haskell}{BUSY} dentro del \mintinline{haskell}{timeout} establecido. Esto se modela con el ciclo con la condición \mintinline{haskell}{emptyCaseLoopCondition} en las líneas 21 a 30.

Si el estado pasa a \mintinline{haskell}{BUSY} antes de que se cumpla el timeout, significa que otro hilo ha leído el valor y está escribiendo el suyo para intercambiar con el hilo actual. Luego el hilo actual lee el valor escrito por el otro, dejando el valor del LFE en \mintinline{haskell}{Nothing} y su estado en \mintinline{haskell}{EMPTY}, y termina la ejecución de la función (líneas 24 a 30).} Para esta función, finalizar la ejecución significa escribir el valor \mintinline{haskell}{False} en la variable \mintinline{haskell}{ret} para no ejecutar nuevamente el ciclo y escribir el valor de retorno en la varible \mintinline{haskell}{res} para después leerla en la última línea (55) de la función.

Si el estado no pasa a \mintinline{haskell}{BUSY} en el ciclo de las líneas 21 a 30, la ejecución saldrá del ciclo en el momento que el tiempo de ejecución supere el \mintinline{haskell}{timeout} establecido. Luego se verificará con una llamada a \mintinline{haskell}{atomCAS} (línea 35) que no haya ocurrido un cambio a último momento en el LFE para que vuelva a su estado inicial \mintinline{haskell}{EMPTY} con valor \mintinline{haskell}{Nothing}. Sin embargo, si hay un cambio significa que otro hilo ha llegado a leer el valor a último momento y se puede realizar el intercambio y finalizar la ejecución como se ve en las líneas 40 a 42.


\item{\mintinline{haskell}{WAITING}: el hilo actual lee el valor escrito y escribe el suyo, luego termina la ejecución de la función. Este comportamiento se ve reflejado en las líneas 45 a 52 de la figura \ref{fig:exchange-io}.}
\item{\mintinline{haskell}{BUSY}: en este caso, ya hay dos hilos realizando un intercambio. Por lo tanto, el hilo actual debe reintentar, volviendo al inicio del ciclo en la línea 9. Este comportamiento se ve reflejado en las líneas 53 y 54 de la figura \ref{fig:exchange-io}, donde el algoritmo simplemente llama a la función \mintinline{haskell}{return ()} para luego entrar a una nueva iteración del ciclo de ejecución.}
\end{itemize}
\newpage

\begin{figure}[!h]
  \centering
  \begin{minted}[linenos,breaklines,fontsize=\footnotesize]{haskell}
exchangeIO :: (Eq a) => LockFreeExchangerIO a -> Maybe a -> Integer -> IO (Maybe a)
exchangeIO lfe myItem timeout = do
  ret <- newIORef True
  res <- newIORef Nothing
  let nanos = timeout * (10 ^ 6) -- timeout unit is millisecs
  timeBound <- systemNanoTime >>= return.((+) nanos)
  stampHolder <- newIORef EMPTY
  whileM_ (readIORef ret) $ do
    timeoutDone <- systemNanoTime >>= return.((<) timeBound)
    if timeoutDone
      then do
        throw TimeoutException
      else do
        yrItem <- getSlotIO (slot lfe) stampHolder
        stamp <- readIORef stampHolder
        case stamp of
          EMPTY -> do
            b <- atomCASIO (slot lfe) (yrItem, EMPTY) (myItem, WAITING)
            if b
              then do
                whileM_ (emptyCaseLoopCondition ret timeBound) $ do
                  yrItem <- getSlotIO (slot lfe) stampHolder
                  stampBusy <- (readIORef stampHolder) >>= return.((==) BUSY)
                  if stampBusy
                    then do
                      writeIORef (slot lfe) (Nothing, EMPTY)
                      writeIORef ret False
                      writeIORef res yrItem
                    else
                      return ()
                breakFromWhile <- readIORef ret >>= return.not
                if breakFromWhile
                  then return ()
                  else do
                    b <- atomCASIO (slot lfe) (myItem, WAITING) (Nothing, EMPTY)
                    if b
                      then do
                        throw TimeoutException
                      else do
                        yrItem <- getSlotIO (slot lfe) stampHolder
                        writeIORef (slot lfe) (Nothing, EMPTY)
                        writeIORef res yrItem
              else do
                return ()
          WAITING -> do
            b <- atomCASIO (slot lfe) (yrItem, WAITING) (myItem, BUSY)
            if b
              then do
                writeIORef ret False
                writeIORef res yrItem
              else
                return ()
          BUSY -> do
            return ()
  readIORef res

  where emptyCaseLoopCondition ret timeBound = do
          timeoutNotDone <- systemNanoTime >>= return.((>) timeBound)
          (readIORef ret) >>= return.((&&) timeoutNotDone)

        systemNanoTime = (getTime Monotonic) >>= return.toNanoSecs
  \end{minted}
  \caption{Código \mintinline{haskell}{exchangeIO}}
  \label{fig:exchange-io}
\end{figure}

El código de la variante \mintinline{haskell}{exchangeSTM} es muy similar al que se muestra en la figura \ref{fig:exchange-io}. La diferencia se encuentra en los renombres de las funciones \mintinline{haskell}{getSlotSTM} y \mintinline{haskell}{atomCASSTM}, junto con la diferencia en cómo los algoritmos escriben la componente \mintinline{haskell}{slot} del LFE. En el caso de IO, vemos en el código que basta con un simple \mintinline{haskell}{writeIORef}, mientras que en STM, esta llamada es reemplazada con la llamada a la función \mintinline{haskell}{writeTVar} dentro de un bloque STM englobado por la función \mintinline{haskell}{atomically} para que los cambios se vean reflejados en el contexto de ejecución.

Para la medición de tiempos se utilizó la función \mintinline{haskell}{getTime} de la librería \mintinline{haskell}{clock} para Haskell \cite{timeMeasuring}. El comportamiento de esta función se explica más adelante en la subsección \ref{time-measuring} del capítulo \ref{chap:experiments} ya que la función es utilizada para medir los tiempos de ejecución de la experimentación.

\subsection{EliminationArray}
Un \mintinline{haskell}{EliminationArray} es una simple estructura de datos que mantiene un arreglo de instancias de \mintinline{haskell}{LockFreeExchanger} y la información pertinente a la duración del \mintinline{haskell}{timeout} que deberá cumplir cada una de las llamadas a \mintinline{haskell}{exchange} que se realicen.
La estructura tiene sólo una función \mintinline{haskell}{visit} que elige aleatoriamente cual de los exchangers de su arreglo utilizar para realizar el intercambio. Luego ejecuta la función \mintinline{haskell}{visit} para el exchanger elegido.

\begin{figure}[!h]
  \centering
  \begin{minted}[linenos,breaklines,fontsize=\footnotesize]{haskell}
data EliminationArrayIO a = EAIO {exchanger :: [LockFreeExchangerIO a], duration :: Integer}

visitIO :: Eq a => EliminationArrayIO a -> Maybe a -> Int -> IO (Maybe a)
visitIO elimArr value range = do
  slot <- randomRIO (0, range)
  exchangeIO ((exchanger elimArr) !! slot) value (duration elimArr)
  \end{minted}
  \caption{Código para \mintinline{haskell}{EliminationArrayIO}}
  \label{fig:elimArr}
\end{figure}

En la figura \ref{fig:elimArr} se muestra el código para un \mintinline{haskell}{EliminationArrayIO}. El código es idéntico al de un \mintinline{haskell}{EliminationArraySTM}, excepto por los renombres correspondientes a las funciones \mintinline{haskell}{exchangeSTM} (línea 6), \mintinline{haskell}{EASTM} (línea 1), y \mintinline{haskell}{visitSTM} junto con el tipado \mintinline{haskell}{LockFreeExchangerSTM} de \mintinline{haskell}{exchanger} en la línea 1.

\subsection{RangePolicy}
Para poder optimizar las probabilidades de lograr un intercambio en ese tiempo de espera. Un EBS cuenta con otra estructura adicional conocida como \mintinline{haskell}{RangePolicy}. Esta estructura tiene como función manejar el rango del \mintinline{haskell}{EliminationArray} en el cual se eligirá el exchanger a utilizar.

La idea de esta política es aumentar el rango a medida que se realizan intercambios con éxito y reducirlo cuando falle.
Al reducir el rango aumenta la probabilidad de que dos hilos elijan la misma posición del arreglo para realizar el intercambio, y al incrementarlo se permite que más intercambios sucedan al mismo tiempo si es que hay suficientes hilos como para cubrir las posiciones del arreglo.

\begin{figure}[!h]
  \centering
  \begin{minted}[linenos,breaklines,fontsize=\footnotesize]{haskell}
data RangePolicy = RgPlcy {maxRange :: IORef Int, currentRange :: IORef Int}

newRangePolicy :: Int -> IO RangePolicy
newRangePolicy maxRange = do
  maxRg <- newIORef maxRange
  currRg <- newIORef 0
  return $ RgPlcy maxRg currRg

recordEliminationSuccess :: RangePolicy -> IO ()
recordEliminationSuccess rp = do
  max <- readIORef $ maxRange rp
  curr <- readIORef $ currentRange rp
  if curr < max
      then writeIORef (currentRange rp) (curr + 1)
      else return ()

recordEliminationTimeout :: RangePolicy -> IO ()
recordEliminationTimeout rp = do
  curr <- readIORef $ currentRange rp
  if curr > 0
      then writeIORef (currentRange rp) (curr - 1)
      else return ()

getRange :: RangePolicy -> IO Int
getRange = readIORef.currentRange
  \end{minted}
  \caption{Código de un \mintinline{haskell}{RangePolicy}}
  \label{fig:range-policy}
\end{figure}

Es importante notar que según \cite{shavit}, cada hilo de ejecución debe tener su propio \mintinline{haskell}{RangePolicy}. Para esto utilizamos la librería \texttt{thread-local-storage} \cite{tls-haskell} de Haskell para permitir que cada hilo tenga su propia instancia de \mintinline{haskell}{RangePolicy}. Esto se muestra en la línea 1 de la figura \ref{fig:ebs-reps} donde la componente \mintinline{haskell}{policy} tiene tipo \mintinline{haskell}{TLS RangePolicy}.

\subsection{Estrategia de backoff del EliminationBackoffStack}
La estructura \mintinline{haskell}{EliminationBackoffStack} (EBS) tiene un comportamiento muy similar a un \mintinline{haskell}{LockFreeStack}. La diferencia se encuentra en las acciones que cada estructura realiza cuando la operación inicial falla, y por lo tanto debe esperar antes de reintentar. En el caso del LFS, el hilo debe esperar un tiempo sin realizar ninguna acción y reintenta, mientras que en un EBS este intenta aprovechar el tiempo de espera para realizar un intercambio con otro hilo que haya fallado su intento.

Luego, en las funciones de apilar y desapilar, se intenta realizar la operación con las mismas funciones que utiliza un LFS, y en el caso de que haya una falla, se obtiene el rango dictado por el \mintinline{haskell}{RangePolicy} y se llama a la función \mintinline{haskell}{visit} del \mintinline{haskell}{EliminationArray} con el rango pasado como argumento.

En el código que se encuentra en las figuras \ref{pushEBSIO-code} y \ref{popEBSIO-code} se puede ver que a cada llamada de la función \mintinline{haskell}{visitIO} se la rodea por la función \mintinline{haskell}{catch} para poder analizar si hubo excepciones arrojadas. De esta manera, el algoritmo determina si el intercambio fue exitoso y modifica la política de rango con las llamadas a las funciones \mintinline{haskell}{recordEliminationTimeout} en caso de fracaso y \mintinline{haskell}{recordEliminationSuccess} en caso de éxito. También se puede ver el uso de las funciones \mintinline{haskell}{tryPushIO} y \mintinline{haskell}{tryPopIO} que son las mismas funciones que se utilizan en un LFS, detalladas en la subsección \ref{sub:lfs-intentos}.

El código para estas funciones es idéntico en la variante STM del EBS, exceptuando los renombres de las funciones y estructuras utilizadas en las distintas estructuras como \mintinline{haskell}{EliminationBackoffStackSTM}, \mintinline{haskell}{newNodeSTM}, \mintinline{haskell}{visitSTM}, \mintinline{haskell}{tryPopSTM}, y \mintinline{haskell}{tryPushSTM}.

\begin{figure}[!h]
  \centering
  \begin{minted}[linenos,breaklines,fontsize=\footnotesize]{haskell}
pushEBSIO :: Eq a => EliminationBackoffStackIO a -> a -> IO ()
pushEBSIO ebs value = do
  ret <- newIORef True
  rangePolicy <- getTLS (policy ebs)

  range <- getRange rangePolicy
  node <- newNodeIO value
  whileM_ (readIORef ret) $ do
    b <- tryPushIO ebs node
    if b
      then writeIORef ret False
      else (catch (tryExchangePush ebs node value range ret rangePolicy) $ \( e :: TimeoutException) -> do
              recordEliminationTimeout rangePolicy)

  where tryExchangePush ebs node value range ret rangePolicy = do
          otherValue <- visitIO (eliminationArray ebs) (Just value) range
          if otherValue == Nothing
            then do
              recordEliminationSuccess rangePolicy
              writeIORef ret False
            else return ()
  \end{minted}
  \caption{Código de \mintinline{haskell}{pushEBSIO}}
  \label{pushEBSIO-code}
\end{figure}

\begin{figure}[!h]
\begin{minted}[linenos,breaklines,fontsize=\footnotesize]{haskell}
popEBSIO :: Eq a => EliminationBackoffStackIO a -> IO a
popEBSIO ebs = do
  res <- newIORef Nothing
  ret <- newIORef True
  rangePolicy <- getTLS (policy ebs)

  range <- getRange rangePolicy
  whileM_ (readIORef ret) $ do
    returnNode <- tryPopIO ebs
    if returnNode /= Null
      then do
        writeIORef res $ Just (val returnNode)
        writeIORef ret False
      else (catch (exchangePop ebs range ret res rangePolicy) $ \( e :: TimeoutException) -> do
          recordEliminationTimeout rangePolicy)
  readIORef res >>= return.fromJust

  where exchangePop ebs range ret res rangePolicy = do
          otherValue <- visitIO (eliminationArray ebs) Nothing range
          case otherValue of
            Just v -> do
              recordEliminationSuccess rangePolicy
              writeIORef res (Just v)
              writeIORef ret False
            otherwise -> return ()
\end{minted}
  \caption{Código de \mintinline{haskell}{popEBSIO}}
  \label{popEBSIO-code}
\end{figure}

\section{StackSTM}\label{sec:stackstm}
La última estructura de datos implementada es \mintinline{haskell}{StackSTM}. La implementación para esta estructura intenta aprovechar lo más posible las herramientas que provee la librería STM de Haskell.

El código para esta implementación se ve en la figura \ref{fig:stack-stm}. Se puede apreciar cómo algoritmos para \mintinline{haskell}{pushStackSTM} y \mintinline{haskell}{popStackSTM} son reducidos a pocas líneas de código comparado a sus otras implementaciones ya que la librería STM permite englobar un bloque de código transaccional dentro de una llamada a la función \mintinline{haskell}{atomically} que asegura que el código será ejecutado atómicamente sin posibilidad que haya conflictos con otros hilos de ejecución.

\begin{figure}[!h]
  \centering
  \begin{minted}[linenos,breaklines,fontsize=\footnotesize]{haskell}
data StackSTM a = ST {top :: TVar (Node a)}

newStackSTM :: IO (StackSTM a)
newStackSTM = atomically (newTVar Null) >>= return.ST

pushStackSTM:: StackSTM a -> a -> IO ()
pushStackSTM st value = do
  node <- newNode value
  atomically $ do
    oldTop <- readTVar $ top st
    writeTVar (next node) oldTop
    writeTVar (top st) node

popStackSTM:: Show a => StackSTM a -> IO a
popStackSTM st = atomically $ do
    resNode <- readTVar $ top st
    case resNode of
      Nd v nxt -> do
        newTop <- readTVar nxt
        writeTVar (top st) newTop
        return v
      Null -> throw EmptyException   
  \end{minted}
  \caption{Implementación completa de \mintinline{haskell}{StackSTM}}
  \label{fig:stack-stm}
\end{figure}