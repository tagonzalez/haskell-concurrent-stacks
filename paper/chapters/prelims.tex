\chapter{Preliminares}\label{chap:prelims}

\section{Programación imperativa en Haskell}

Al ser un lenguaje de programación funcional, no es común relacionar los conceptos de lenguajes imperativos con Haskell. No obstante, el lenguaje provee funcionalidades interesantes para utilizar el lenguaje de manera similar a un lenguaje imperativo sin dejar de ser funcional.
Esto se logra con el uso de la mónada IO y un azúcar sintáctico conocido como \emph{\mintinline{haskell}{do} notation}.

\subsection{Mónadas}
Una mónada es un tipo de datos que permite representar formas específicas de computaciones y capturar efectos. Estas tareas incluyen manejo de excepciones, mantener un estado, generar de números aleatorios, y realizar acciones de entrada y salida.

El comportamiento de una mónada se define por dos funciones: \mintinline{haskell}{return} y el operador \emph{bind} \mintinline{haskell}{(>>=)} que se detallan en la Fig. \ref{fig:io-functions}.

\begin{figure}[!h]
  \centering
\begin{minted}[linenos,breaklines,fontsize=\footnotesize]{haskell}
return :: a -> m a
(>>=)  :: m a -> (a -> m b) -> m b
\end{minted}
  \caption{Tipado de las funciones de una mónada \mintinline{haskell}{m}}
  \label{fig:io-functions}
\end{figure}

La función \mintinline{haskell}{return} se encarga de convertir un valor pasado por parámetro a un valor mónadico mientras que la función \emph{bind} se ocupa de componer dos acciones monádicas, pasando el resultado de la primera acción (primer parámetro) a la próxima. Más información sobre estas funciones se encuentra detallada en la documentación de la librería Control.Monad \cite{control-monad}.

Una de las mónadas más utilizadas es la mónada IO. Esta permite implementar acciones de entrada/salida como, por ejemplo, imprimir por pantalla o generar números aleatorios.



En la Fig. \ref{fig:functional-randomSum} se muestra el código para una función \mintinline{haskell}{randomSum} que genera dos números aleatoriamente e imprime la suma por pantalla. Para la generación de números aleatorios se utiliza la función \mintinline{haskell}{randomIO} de la librería \texttt{random} de Haskell \cite{random}.

El código de la función consiste en generar un número aleatorio con una llamada a la función \mintinline{haskell}{randomInt} y pasar el resultado a una segunda función utilizando el operador \mintinline{haskell}{(>>=)}. La segunda función tomará el resultado de \mintinline{haskell}{randomInt} como parámetro \texttt{rand1}, realizará una nueva llamada a \mintinline{haskell}{randomInt} y pasa el resultado (\texttt{rand2}) a una última función que imprime el resultado de la suma de \texttt{rand1} y \texttt{rand2}.

\begin{figure}[!h]
  \centering
\begin{minted}[linenos,breaklines,fontsize=\footnotesize]{haskell}
randomInt :: IO Int
randomInt = randomIO :: IO Int

printRandomSum :: IO ()
printRandomSum = randomInt >>= (\ rand1 -> randomInt >>= (\ rand2 -> print (rand1 + rand2)))
\end{minted}
  \caption{Ejemplo de acción monádica \mintinline{haskell}{printRandomSum}}
  \label{fig:functional-randomSum}
\end{figure}

Como alternativa para la implementación de acciones monádicas, el lenguaje Haskell  provee lo que se conoce como \emph{\mintinline{haskell}{do}-notation} que permite escribir código siguiendo un estilo imperativo que luego sería interpretado como una composición de funciones. Esta notación es simplemente un azúcar sintáctico para facilitar la implementación de acciones monádicas como la presentada en la Fig. \ref{fig:functional-randomSum}.
En la Fig. \ref{fig:do-randomSum} se puede apreciar la implementación de esta función utilizando \emph{\mintinline{haskell}{do}-notation}.

Se puede apreciar cómo la implementación de esta función se facilita gracias a la notación \mintinline{haskell}{do} en la Fig. \ref{fig:do-randomSum}.

\begin{figure}[!h]
  \centering
\begin{minted}[linenos,breaklines,fontsize=\footnotesize]{haskell}
randomInt :: IO Int
randomInt = randomIO :: IO Int

printRandomSum :: IO ()
printRandomSum = do
  rand1 <- randomInt
  rand2 <- randomInt
  print (rand1 + rand2)
\end{minted}
  \caption{Código de \mintinline{haskell}{printRandomSum} reescrito con notación \mintinline{haskell}{do}}
  \label{fig:do-randomSum}
\end{figure}

En \cite{do-notation} se describe en detalle cómo este azúcar sintáctico logra traducir lo que se escribe como código imperativo a código funcional a nivel de compilación.

\section{Algoritmos optimistas}\label{sec:lockfree}
Los algoritmos optimistas actúan sobre un recurso compartido bajo la suposición de que no ocurrirán conflictos entre los distintos hilos de ejecución y, por lo tanto no adquieren locks mientras acceden al recurso. Sin embargo, requiren verificar que ningun otro hilo ha interferido antes de consolidar sus modificaciones. Para ello, se basan en el uso de acciones atomicas, tales como CAS para comparar estado y modificarlo simultaneamente.
En el caso que la llamada a CAS falle, el algoritmo vuelve a ejecutarse hasta que el recurso sea modificado con éxito.

Esta clase de algoritmos permiten que los distintos hilos de ejecución no deban frenar su ejecución para esperar la liberación de un recurso, como es el caso en algoritmos que utilizan algún tipo de lock sobre un recurso.
De esta manera, uno puede obtener mejores tiempos de ejecución ya que los tiempos de espera son eliminados.

En la Fig. \ref{fig:lockfree-counter-example} se presenta, a modo de ejemplo, una estructura de datos \mintinline{haskell}{Counter} representada por una referencia a un entero y un algoritmo optimista \mintinline{haskell}{inc} para incrementar su valor.

\begin{figure}[!h]
\begin{minted}[linenos,breaklines,fontsize=\footnotesize]{haskell}
data Counter = Ref Int

inc :: Counter -> IO ()
inc counter = do
    loop <- newRef True
    whileM_ (readRef loop) $ do
        oldValue <- readRef counter
        let newValue = oldValue + 1
        success <- atomCAS counter oldValue newValue
        if success
            then writeRef loop False
            else return ()
\end{minted}
\caption{Algoritmo optimista para incrementar un contador}
\label{fig:lockfree-counter-example}
\end{figure}

En el algoritmo, primero se declara una referencia booleana \texttt{loop} para mantener un ciclo con la función de Haskell \texttt{whileM\_}.
En el ciclo se realiza una lectura del valor del contador, guardando el resultado en la variable \texttt{oldValue}, para luego realizar la llamada a la función \texttt{atomCAS}.

Esta llamada realiza una nueva lectura del valor del contador y lo compara con \texttt{oldValue} para determinar si es correcto modificar el valor.
Si el valor leído coincide con el de \texttt{oldValue} se modifica el valor del contador con el valor de \texttt{newValue}, retornando \mintinline{haskell}{True} como resultado de la función.
Si no, significa que un hilo de ejecución distinto al actual logró modificar el valor del contador luego de que el hilo actual lea el valor en la línea 7 del código.
En este caso no se realizan cambios y la llamada a \texttt{atomCAS} retornará \mintinline{haskell}{False}.
Es importante notar que todas las acciones de la función \texttt{atomCAS} son realizadas de manera atómica.

Según el valor de retorno, el algoritmo de la función \mintinline{haskell}{inc} terminará la ejecución modificando el valor de la referencia \texttt{loop} a \mintinline{haskell}{False}, deteniendo el ciclo, o volverá a iniciar el ciclo como se ve en el condicional \mintinline{haskell}{if} de la línea 10.
La línea 12 de la Fig. \ref{fig:lockfree-counter-example} que llama a la función \texttt{return} con parámetro \mintinline{haskell}{()} no realiza ninguna operación y simplemente permite volver a iniciar el ciclo.

En este trabajo se utiliza este esquema de algoritmos en varias de las implementaciones. También se presentan variantes para la implementación de algoritmos de este estilo ya que es posible definirlos utilizando distintas primitivas de sincronización para implementar distintas versiones de \texttt{atomCAS}.
Estas versiones se detallan en la sección \ref{sec:primitives}.

\section{Primitivas de sincronización} \label{sec:primitives}
El lenguaje de programación Haskell ofrece distintas primitivas de sincronización para poder manejar recursos compartidos entre distintos hilos de ejecución.
En este trabajo se utilizan los tipos de datos \mintinline{haskell}{IORef} y \mintinline{haskell}{TVar} de la librería \mintinline{haskell}{STM} para las implementaciones que se detallan en el próximo capítulo.
Estos tipos de datos son distintas maneras de referenciar objetos mutables, similares a punteros en otros lenguajes de programación, y proveen funciones para ser modificadas de manera atómica.

\subsection{IORef}\label{sub:ioref}
El tipo de datos \mintinline{haskell}{IORef} es comunmente utilizado en programas Haskell como manera de referenciar objetos mutables.
En la Fig. \ref{fig:ioref-interface} se detallan las funciones para crear, leer, escribir y modificar atómicamente los valores contenidos en una referencia \mintinline{haskell}{IORef}.

\begin{figure}[!h]
\centering
\begin{minted}[linenos,breaklines,fontsize=\footnotesize]{haskell}
newIORef   :: a -> IO (IORef a)
readIORef  :: IORef a -> IO a
writeIORef :: IORef a -> a -> IO a
atomicModifyIORef :: IORef a -> (a -> (a,b)) -> IO b
\end{minted}
\caption{Funciones para \mintinline{haskell}{IORef}}
\label{fig:ioref-interface}
\end{figure}

La función \mintinline{haskell}{atomicModifyIORef} de la última línea de la Fig. \ref{fig:ioref-interface} es la única función que realiza una modificación sobre una referencia del tipo \mintinline{haskell}{IORef} de manera atómica.
La función recibe como parámetros una referencia a modificar y una función que cumple dos tareas: modificar el valor contenido en la referencia y luego retornar un valor en función del valor previo contenido en la referencia como resultado de \texttt{atomicModifyIORef}. El código que presenta en la Fig. \ref{fig:atomicModifyIORef-behavior}, previamente presentado en \cite{linked-list}, muestra un comportamiento equivalente al de \mintinline{haskell}{atomicModifyIORef}.

\begin{figure}[!h]
\centering
\begin{minted}[linenos,breaklines,fontsize=\footnotesize]{haskell}
atomicModifyIORef r f = do
   a <- readIORef r
   let p = f a
   writeIORef r (fst p)
   return (snd p)
\end{minted}
\caption{Comportamiento de la función \mintinline{haskell}{atomicModifyIORef}}
\label{fig:atomicModifyIORef-behavior}
\end{figure}

Para implementar una versión de \texttt{atomCAS} es crucial contar con una función como \mintinline{haskell}{atomicModifyIORef} ya que es la única función que permite manipular referencias de tipo \mintinline{haskell}{IORef} de manera atómica.
Generalmente, una función de CAS como \texttt{atomCAS} debe recibir tres parámetros: la referencia a modificar, el valor previamente leído a comparar con el valor actual de la referencia atómicamente, y el valor nuevo a guardar en la referencia.
La función debe retornar un booleano que indica si la comparación entre el valor leído y el valor actual de la referencia es exitosa.

En este caso, para aplicar \texttt{atomCAS} sobre una referencia de tipo \mintinline{haskell}{IORef}, utilizaremos la función de \mintinline{haskell}{atomicModifyIORef} pasándole como segundo parámetro una función que compare el valor leído con el actual y retorne una tupla con el valor a guardar en la referencia y el booleano resultante de la comparación.
Si la operación es exitosa, se guarda el valor nuevo en la referencia.
En caso contrario, se vuelve a guardar el valor actual.
Este comportamiento se refleja en el código de la Fig. \ref{fig:atomcasio}, también presentado en \cite{linked-list}.

\begin{figure}[!h]
\begin{minted}[linenos,breaklines,fontsize=\footnotesize]{haskell}
atomCASIO :: Eq a => IORef a -> a -> a -> IO Bool
atomCASIO ptr old new =
  atomicModifyIORef ptr (\ cur -> if cur == old
                                  then (new, True)
                                  else (cur, False))
\end{minted}
\caption{\mintinline{haskell}{atomCAS} utilizando IORef}
\label{fig:atomcasio}
\end{figure}

Implementada esta función, se puede reescribir el código del contador como se muestra en la Fig. \ref{fig:lockfree-counter-example-ioref}, utilizando \mintinline{haskell}{IORef} como referencia y las funciones \mintinline{haskell}{readIORef}, \mintinline{haskell}{writeIORef}, y \mintinline{haskell}{atomCASIO}.

\begin{figure}[!h]
\begin{minted}[linenos,breaklines,fontsize=\footnotesize]{haskell}
data Counter = IORef Int

inc :: Counter -> IO ()
inc counter = do
    loop <- newIORef True
    whileM_ (readIORef loop) $ do
        oldValue <- readIORef counter
        let newValue = oldValue + 1
        success <- atomCASIO counter oldValue newValue
        if success
            then writeIORef loop False
            else return ()
\end{minted}
\caption{Algoritmo optimista para incrementar un contador utilizando \mintinline{haskell}{IORef}}
\label{fig:lockfree-counter-example-ioref}
\end{figure}

\subsection{STM}\label{sub:stm}
La librería STM de Haskell permite al programador componer distintas funciones a ser ejecutadas de manera atómica.
Para lograr esto, la librería provee el tipo de mónada STM para distinguir las funciones que se realizan dentro de esta mónada con la mónada IO y el tipo de referencia \mintinline{haskell}{TVar}.
Es decir, las funciones de la mónada STM sólo pueden componerse con funciones de la misma mónada o funciones puras (que no causan efectos al contexto de ejecución). Una vez realizada la composición de funciones, se llama a una función \texttt{atomically} para que las acciones realizadas dentro de la mónada STM sean realizadas atómicamente y sus efectos sean visibles al nível de la mónada IO.

Intuitivamente, uno puede interpretar a la composición de funciones STM como una transacción cuyos cambios realizados sólo pueden ser visibles una vez que se completa.
Dentro de una transacción, se debe utilizar un tipo de referencia distinto de \mintinline{haskell}{IORef}, ya que sus funciones no pueden invocarse dentro de la mónada STM. Para esto se tiene el tipo de variable transaccional \mintinline{haskell}{TVar} y funciones para su escritura y lectura.
Las funciones provistas por la librería STM en la Fig. \ref{fig:stm-interface}.

\begin{figure}[!h]
\begin{minted}[linenos,breaklines,fontsize=\footnotesize]{haskell}
atomically :: STM a -> IO a
newTVar :: TVar a
readTVar :: a -> STM (TVar a)
writeTVar :: TVar a -> STM a -> a -> STM ()
\end{minted}
\caption{Funciones de la librería STM}
\label{fig:stm-interface}
\end{figure}

La librería STM también provee unas funciones \texttt{retry} y \texttt{orElse} que permiten manipular el flujo de una transacción para que se realice un reintento o ejecutar una transacción en vez de otra en caso de que no sea exitosa.
Sin embargo, no fueron necesarias para las implementaciones que se realizaron en este trabajo que se encuentran detalladas en el próximo capítulo.

Dadas estas herramientas, se puede realizar una nueva implementación de la función \texttt{atomCAS} sobre STM.
El código de la función también fue presentado en \cite{linked-list}.

\begin{figure}[!h]
\begin{minted}[linenos,breaklines,fontsize=\footnotesize]{haskell}
atomCASSTM :: Eq a => TVar a -> a -> a -> IO Bool
atomCASSTM ptr old new = atomically $ do
  cur <- readTVar ptr
  if cur == old
  then do
    writeTVar ptr new
    return True
  else return False
\end{minted}
\caption{\mintinline{haskell}{atomCAS} utilizando STM}
\end{figure}

Dadas estas herramientas, uno puede implementar una versión del contador sobre STM utilizando \mintinline{haskell}{TVar} como tipo de referencia y utilizando las funciones \mintinline{haskell}{atomCASSTM}, y \mintinline{haskell}{readTVar}.

\begin{figure}[!h]
\begin{minted}[linenos,breaklines,fontsize=\footnotesize]{haskell}
data Counter = TVar Int

inc :: Counter -> IO ()
inc counter = do
    loop <- newIORef True
    whileM_ (readIORef loop) $ do
        oldValue <- atomically $ readTVar counter
        let newValue = oldValue + 1
        success <- atomCASSTM counter oldValue newValue
        if success
            then writeIORef loop False
            else return ()
\end{minted}
\caption{Algoritmo optimista para incrementar un contador utilizando STM}
\label{fig:lockfree-counter-example-stm}
\end{figure}

Es importante notar que en el código de la Fig. \ref{fig:lockfree-counter-example-stm} la mónada continúa siendo IO ya que la llamada a la función \texttt{readTVar} es invocada dentro de la función \texttt{atomically}.

Además de poder implementar variantes de algoritmos optimistas, podemos implementar nuestro contador de una manera distinta utilizando las funciones STM como se muestra en la Fig. \ref{fig:counter-example-stm}.

\begin{figure}[!h]
\begin{minted}[linenos,breaklines,fontsize=\footnotesize]{haskell}
data Counter = TVar Int

inc :: Counter -> IO ()
inc counter = atomically $ do
    oldValue <- readTVar counter
    let newValue = oldValue + 1
    writeTVar counter newValue
\end{minted}
\caption{Algoritmo STM para incrementar un contador}
\label{fig:counter-example-stm}
\end{figure}

Se puede apreciar cómo este estilo de algoritmos resultan más intuitivos para interpretar e implementar que sus versiones optimistas gracias a que STM nos permite componer las funciones que necesitamos que sean ejecutadas atómicamente en vez de realizar llamadas a CAS y realizar reintentos en caso de que falle.
