\chapter{Conclusión}

Entre las cinco versiones implementadas, uno puede concluir que las implementaciones de EBS no son las más adecuadas en términos de performance para ser utilizados, especialmente la versión IO que presentó varias instancias de anomalías en sus resultados. Este resultado soprende dado que la estructura EBS fue propuesta como una alternativa de mayor escalabilidad que LFS. En nuestros resultados para los experimentos que variaron la cantidad de hilos, se encontró que las implementaciones LFS mantuvieron tiempos de ejecución bajos en comparación a las implementaciones EBS. En cuanto a la implementación STM, esta resulta ser la mejor opción cuando la cantidad de hilos es baja, acercándose a tiempos de ejecución parecidos a LFS a medida que se agregan hilos. Otra observación importante es que las implementaciones de LFS no son afectadas por el incremento en la cantidad de núcleos de ejecución como el resto de las implementaciones.

Los experimentos realizados dan lugar a analizar distintos casos de uso. Por ejemplo, si el caso de uso para una pila concurrente es un escenario donde la cantidad de hilos de ejecución es constante y lo que varía sean las operaciones que realizan los hilos uno se encuentra en una situación parecida al experimento \mintinline{haskell}{pushPercentages} y puede optar por la implementación que mejores resultados presentó. En este caso, la implementación STM es una opción clara dado que obtuvo los menores tiempos de ejecución. Si el caso de uso para la pila concurrente requiere de mayor estabilidad en los tiempos de ejecución a con una alta variación en la cantidad de hilos que ejecutan sobre la pila, cualquiera de las implementaciones LFS son buenos candidatos a elegir para el escenario. La implementación sobre STM también es una buena opción, pero uno debería tener cuidado si la cantidad de hilos se vuelve muy grande ya que la tendencia de los resultados muestra que la implementación puede llegar a tiempos de ejecución mayores a los de LFS.

Es importante analizar también la complejidad de las estructuras al momento de implementarlas. Las implementaciones de EBS son sin duda las más complejas dadas las estructuras auxiliares de las que depende como el exchanger y el arreglo de eliminación. Por el otro lado, la implementación de STM es la más fácil de lograr una vez obtenido conocimiento sobre el funcionamiento de la librería STM.

